\documentclass[letterpaper,11pt,twocolumn]{article}
\usepackage{fullpage}
\usepackage{color}
\usepackage{graphicx}
\usepackage{epsfig}
\usepackage{amsthm}
\usepackage{latexsym}
\usepackage{amssymb}
\usepackage{amsmath}

\newcommand{\newfontobj}[2]{
  \newcommand{#1}[1]{
    \expandafter\def\csname##1\endcsname{{#2 ##1}}}}

\newfontobj{\class}{\rm} % Typeset Classes in roman font

% Some standard classes (use in only mathmode)
% Usage example: $\P \subseteq \NP$ and we believe that $\NP$ is not equal to $\P$.

\class{PSPACE}
\class{L}
\class{BPL}
\class{RL}
\class{NC}
\class{ZPL}
\class{NPSPACE}
\class{ASPACE}
\class{NL}
\class{EXP}
\class{NEXP}
\class{coNEXP}
\class{NE}
\class{E}
\class{AM}
\class{MA}
\class{NP}
\class{DNP}
\class{UP}
\class{P}
\class{RP}
\class{BPP}
\class{ZPP}
\class{EXPSPACE}
\class{coNP}
\class{coRP}
\class{coAM}
\class{PH}
\class{IP}
\class{PCP}
\class{MIP}
\class{SE}

% operator classes.
\class{BP}

% these commands should be used in math mode - $ $
\newcommand{\SHARPP}{{\#\rm{P}}}
\newcommand{\PARITYP}{{\oplus\rm{P}}}

% math operators...
\DeclareMathOperator{\poly}{poly}
\DeclareMathOperator{\Majority}{Majority}
\DeclareMathOperator{\quasipoly}{quasi-poly}
\DeclareMathOperator{\polylog}{poly-log}
\DeclareMathOperator{\superpoly}{super-poly}
\DeclareMathOperator{\DTISP}{DTISP}
\DeclareMathOperator{\DSPACE}{DSPACE}
\DeclareMathOperator{\DTIME}{DTIME}
\DeclareMathOperator{\NSPACE}{NSPACE}
\DeclareMathOperator{\NTIME}{NTIME}
\DeclareMathOperator{\BPTIME}{BPTIME}
\DeclareMathOperator{\RTIME}{RTIME}
\DeclareMathOperator{\ZPTIME}{ZPTIME}
\DeclareMathOperator{\BPSPACE}{BPSPACE}
\DeclareMathOperator{\RSPACE}{RSPACE}
\DeclareMathOperator{\ZPSPACE}{ZPSPACE}
\DeclareMathOperator{\med}{med}


% Complexity class
\newcommand{\CC}{\mathcal{C}}


% add DRAFT to your document %
\newcommand{\draft}[0]{
\begin{center}
	{\bf \Large {\sc DRAFT} }
\end{center}
}

% example environment
\newenvironment{example}
{\smallskip \noindent \emph{Example:}}
{\hfill $\boxtimes$ \smallskip}

% some theorem environments
\newtheorem{conjecture}{Conjecture}
\newtheorem{theorem}{Theorem}
\newtheorem{proposition}{Proposition}
\newtheorem{claim}{Claim}
\newtheorem{lemma}{Lemma}
\newtheorem{corollary}{Corollary}
\newtheorem{definition}{Definition} % Use this for non-trivial
	% definitions.

% currently not used %
\newtheorem{exercise}{Exercise}
\newtheoremstyle{example}{\topsep}{\topsep}%
     {\normalfont \small}   % Body font
     {}    % Indent amount (empty = no indent, \parindent = para indent)
     {\bfseries}     % Thm head font
     {}%           Punctuation after thm head
     {\topsep}%     Space after thm head
     {}%         Thm head spec    \theoremstyle{example}
\theoremstyle{example}
%\newtheorem{example}{Example}

\usepackage{multicol}
\usepackage{fullpage}
\usepackage{url}
\usepackage[top=1in, bottom=1in, left=1in, right=1in]{geometry}
\usepackage{hyperref}
\usepackage{multirow}
\usepackage{framed}
\usepackage{enumerate}
\pagenumbering{arabic}
\setlength{\columnsep}{0.25in}
\usepackage{paralist}
\let\itemize\compactitem

\def\bfw{\mathbf w}
\def\bfW{\mathbf W}
\def\bfD{\mathbf D}
\def\bfY{\mathbf Y}
\def\bfX{\mathbf X}
\def\bfU{\mathbf U}
\def\bfV{\mathbf V}
\def\bfS{\mathbf \Sigma}
\def\bfx{\mathbf x}
\def\R{\mathbb R}
\def\F{\mathrm F}

\title{\bf{Twitter hashtag implication using various machine learning methods}}
\author{
  Chang Wang\\
  Department of Computer Sciences\\
  \texttt{wych92@cs.wisc.edu}
  \and
  Lichao Yin\\
  Department of Computer Sciences\\
  \texttt{lyin28@wisc.edu}
  \and
  Chaowen Yu\\
  Department of Computer Sciences\\
  \texttt{ycw@cs.wisc.edu}
  \and
  Biao Zhang\\
  Department of Computer Sciences\\
  \texttt{bzhang263@wisc.edu}
}
\date{May 9, 2015}
\begin{document}
\twocolumn[
\maketitle
]
\begin{abstract}
\textbf{keywords}: text classification
\end{abstract}


\section{Introduction}
\label{sec:intro}



\section{Data Preparation}
\label{sec:data}
\subsection{Data Collection}
\subsection{Data Preprocessing}
After above steps, we get 5000 tweets with corresponding hashtags like this:
\begin{framed}
The House of Black and White - The Wars to Come - \#gameofthrones \#got \#got5 \#e2s5  \#hbo $\backslash$u2026 https://t.co/6zrQXilrZe
\end{framed}
It is obvious that \emph{http://}, \emph{$\backslash$u2026} and \emph{\#gameofthrones} do little contribution to classification. So we find them out through 5000 tweets and remove them. For convinience, we then remove all non-alphanumeric characters and turn all letters to lowercase. Thus the example tweet turns to be:
\begin{framed}
the house of black and white the wars to come
\end{framed}
Because computer usually fails to capture semantic information from human languages, it is common to represent a text as a vector corresponding to the terms(usually terms are words) that appear in the text. For our problem, we will use bag-of-words model. Here we define the following terms as denoted as \cite{blei2003latent}:
\begin{itemize}
\item A \emph{term} is the basic unit of discrete data denoted by $w$, which is usually a word appearing in a tweet.
\item A \emph{tweet} contains $N$ terms, denoted by $\bfw = (w_1, w_2, \dots, w_N)$.
\item A \emph{corpus} is a collection of $M$ tweets($M=5000$). The tweets texts are denoted by $\bfD = \{\bfw_1, \bfw_2, \dots, \bfw_M\}$, and their corresponding tag labels are denoted by $\bfY = \{y_1, y_2, \dots, y_M\}$ where $y_i \in \{1, 2, 3, 4, 5\}, i = 1, 2, \dots, M$. For our problem, 1 stands for the hashtag "", 2 stands for the hashtag "", 3 stands for the hashtag "", 4 stands for the hashtag "", and 5 stands for the hashtag "".
\item A \emph{dictionary} is a collection of all $V$ unique terms appearing in the corpus denoted by $\bfW = \{w_1^*, w_2^*, \dots, w_V^*\}$.
\end{itemize}
Let $\bfX$ be a term-tweet matrix. This matrix has $M$ rows(one row for each tweets) and $V$ columns(one column for each unique term in dictionary). The element $x_{ij}$ in $\bfX$ indicates whether the $j$-th term appears in the $i$-th tweet. If it does appear, $x_{ij} = 1$; otherwise $x_{ij} = 0$.\\
In general, because most tweets will just use a small part of terms in the dictionary, a lot of $x_{ij}$ will be zero. So the matrix $\bfX$ is sparse. However, a sparse $\bfX$ wastes too much space and processes much slower in high dimension. As we know, there are some high-frequency terms appearing in all documents, such as function words(e.g., \emph{a}, \emph{or}, \emph{on}) and pronouns(e.g., \emph{which}, \emph{it}, \emph{those}). These terms have little contribution to representing different documents because of relatively low information content, so we shall remove them from our dictionary. \cite{salton1971smart} developed a SMART system with a popular list of more than 500 common terms, which called \emph{stop-word list} and can be used for our model.\\
After removing \emph{stop-words}, our final $\bfX$ has 5000 rows for tweets and 9791 columns for unique terms, and $\bfY$ has 5000 rows for each tweet's tag label.
\subsection{Cross Validation}
We will use stratified sampling for cross validation.\\
Since we have 1000 tweets for every 5 hashtags. We equally split each 1000 hashtag tweets into 5 parts. And each time we select and combine 4 parts from every hashtags for training and the rest part for testing.\\
Therefore, we have 5 datasets in all. Each dataset has training matricies: $\bfX_{training} \in \R^{4000 \times 9791}$ and $\bfY_{training} \in \{1, 2, 3, 4, 5\}^{4000}$, and testing matrices $\bfX_{testing} \in \R^{1000 \times 9791}$ and $\bfY_{testing} \in \{1, 2, 3, 4, 5\}^{1000}$.

\section{Implementation}
\label{sec:impl}
\subsection{Decision Tree}
To lean a multi-class decision tree, we use the implementation from scikit-learn 1.6.1. scikit-learn uses an optimised version of the CART algorithm.\\
\subsection{Random Forest}

\subsection{Neural Network}
\subsection{Naive Bayes}
\subsection{Support Vector Machine}
\subsection{Logistic Regression Model}
\subsection{$k$ Nearest Neighbors}


\section{Results}
\label{sec:eval}
% \begin{figure}[!htb]
% \centering
% \epsfig{file=evaluation.eps, width=3.4in}
% \caption{Boot-up time for various configurations}
% \label{figure:eval}
% \end{figure}

% \begin{figure*}[!htb]
% \centering
% \epsfig{file=latencies.eps, width=6.8in}
% \caption{Latencies for various IPC methods on MicroswiftOS and archlinux}
% \label{figure:latency}
% \end{figure*}

% \begin{figure*}[!htb]
% \centering
% \epsfig{file=throughput.eps, width=6.8in}
% \caption{Throughputs for various IPC methods on MicroswiftOS and archlinux}
% \label{figure:throughput}
% \end{figure*}

\section{Related Works}
\label{sec:related}
\section{Future Work}
\label{sec:future}

\section{Acknowledgement}
We would like to thank Michael Swift, Sanketh Nalli and Zhaoyu Luo for their helpful advices and comments. 



% \bibliographystyle{abbrv}
\bibliographystyle{plain}
\bibliography{references}

\end{document}

% \begin{table}[!htb]
% \centering
% \begin{tabular}{|c||c|c|c|}
% \hline
% \multirow{3}{*}{\textbf{Benchmark}} & \textbf{Additional} & \textbf{Total} &\multirow{3}{*}{\textbf{Percentage}}\\ 
% &\textbf{Memory} &\textbf{Memory}&\\
% &\textbf{Accesses} &\textbf{Accesses} &\\ \hline \hline
% aster & 175481 & 2404970 & 0.0730\\ \hline 
% bzip2 & 1067380 & 9188892 & 0.1162\\ \hline 
% lbm   & 130990 & 5003775 & 0.0262\\ \hline 
% libquantum   & 54796 & 3092800 & 0.0177\\ \hline 
% milc  & 136746 & 5575859 & 0.0245\\ \hline 
% omnetpp  & 107165 & 287076 & 0.3733\\ \hline 
% \textbf{Average} & \textbf{-} &\textbf{- } &\textbf{0.1051}\\ \hline
% \end{tabular}
% \caption{Additional memory accesses for storing reuse data in a 1MB L2 reuse cache during a simulation period of 300M instructions}
% \label{table:additionalaccess}
% \end{table}

% \begin{figure}[!htb]
% \centering
% \epsfig{file=ReuseStruct.eps, height =2in, width =2in}
% \caption{Structure of our implementation for the reuse cache}
% \label{reuse:struct}
% \end{figure}

% \begin{table}[!htb]
% \centering
% \begin{tabular}{|c||c|c|c|}
% \hline
% \multirow{3}{*}{\textbf{benchmark}} & \textbf{memory} & \textbf{total} & \multirow{3}{*}{\textbf{percentage}}\\
% &\textbf{write} &\textbf{memory} &\\
% & \textbf{miss times} & \textbf{access times} & \\ \hline\hline
% bzip2 & 778893 & 16791412 & 0.0464\\ \hline 
% aster & 83555 & 4487131 & 0.0186\\ \hline 
% libquantum & 35243 & 6091860 & 0.0058\\ \hline 
% lbm & 72942 & 9862259 & 0.0074\\ \hline 
% milc & 69407 & 10820712 & 0.0064\\ \hline 
% omnetpp & 83508 & 416317 & 0.2006 \\ \hline

% \textbf{Average} &\textbf{-} &\textbf{-} &\textbf{0.0475}\\

% \hline
% \end{tabular}
% \caption{Memory write miss times in a 1MB reuse cache during a simulation period of 300M instructions}
% \label{table:writemiss}
% \end{table}



% \begin{table}[!htb]
% \centering
% \begin{tabular}{|l||r|}
% \hline
% L1 data cache size & 64KB\\
% \hline
% L1 data cache associativity & 4-way\\
% \hline
% L1 instruction cache size & 32KB\\
% \hline
% L1 instruction cache associativity & 4-way\\
% \hline
% L2 cache size & 1MB\\
% \hline
% L2 cache associativity & 8-way\\
% \hline
% Cache line size & 64B\\
% \hline
% Single memory size & 512 MB\\
% \hline
% \end{tabular}
% \caption{System configurations for baseline performance}
% \label{table:baselineconfig}
% \end{table}


% \begin{figure*}[!htb]
% \centering
% \epsfig{file=rc248.eps, width=7in}
% \caption{Relative performace of reuse cache under varying data array sizes. The data array sizes for ratio 2, 4, 8 are 512KB, 256KB, 128KB, respectively.}
% \label{fig:rc248}
% \end{figure*}



% \begin{figure}[!htb]
% \centering
% \epsfig{file=bzip2live.eps, width=3.4in}
% \caption{bzip2 percentage of live lines during execution of workload. Sampling is once per 100K data accesses.}
% \label{fig:bzip2live}
% \end{figure}

% \begin{figure*}[!htb]
% \centering
% \epsfig{file=addarray.eps, width=7in}
% \caption{Relative performace of reuse cache under varying additional block array sizes. The additional block array sizes for ratio 16, 32, 64 are 64KB, 32KB, 16KB, respectively. No additional block array means a regular reuse cache design. All statistics are obtained when reuse cache tag/data ratio is 2.}
% \label{fig:addarray}
% \end{figure*}


% \begin{table*}[!htb]
% \centering
% \begin{tabular}{|c||c|c|c|c|c|c|}
% \hline
% \multirow{2}{*}{\textbf{Benchmark}} & 
% \multicolumn{2}{c|}{\textbf{Ratio 2}} & \multicolumn{2}{c}{\textbf{Ratio 4}} & \multicolumn{2}{|c|}{\textbf{Ratio 8}}\\
% \cline{2-7}
% & \textbf{Speedup} & \textbf{Miss Rate} & \textbf{Speedup} & \textbf{Miss Rate} & \textbf{Speedup} & \textbf{Miss Rate} \\ \hline
% astar & 0.983 & 1.046 & 0.984 & 1.042 & 0.972 & 1.069\\ \hline 
% bwaves & 0.923 & 1.064 & 0.935 & 1.079 & 0.945 & 1.067\\ \hline 
% bzip2 & 0.858 & 1.114 & 0.833 & 1.112 & 0.820 & 1.154\\ \hline 
% lbm & 0.966 & 1.010 & 0.974 & 0.981 & 0.973 & 0.982\\ \hline 
% libquantum & 0.957 & 1.015 & 0.964 & 1.020 & 0.979 & 1.045\\ \hline 
% mcf & 0.876 & 1.700 & 0.833 & 2.030 & 0.796 & 2.383\\ \hline 
% milc & 0.988 & 1.016 & 0.989 & 1.021 & 0.992 & 1.014\\ \hline 
% omnetpp & 0.973 & 1.689 & 0.940 & 3.202 & 0.848 & 8.054\\ \hline 
% \textbf{Average} & \textbf{0.939} &\textbf{1.070} &\textbf{0.930} & \textbf{1.091} &\textbf{0.913} &\textbf{1.148}\\ \hline
% \end{tabular}
% \caption{Speedup and miss rate of workloads relative to baseline. Ths miss rate is a relative ratio compared with baseline miss rate.}
% \label{table:missrate}
% \end{table*}

